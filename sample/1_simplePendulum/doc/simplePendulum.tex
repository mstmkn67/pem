\documentclass[a4paper,11pt]{jarticle}
\usepackage{graphicx}% Include figure files
\usepackage{dcolumn}% Align table columns on decimal point
\usepackage{bm}% bold math
\usepackage{amssymb}
\usepackage{amsmath}

\def\vecR {\bm {\mathcal {R} } }
\def\R  {\mathcal {R} }
\pagestyle{plain}
\title{単純な振り子}
\author{牧野真人}
\date{\number\year 年\number\month 月\number\day 日}
\begin{document}
\maketitle
%%%%%%%%%%%%
\section{理論}
長さ$l$で重力場$g$にある平面内にある振り子は、重力と振り子がなす角度を$\theta$とすると次の方程式で書ける。
\begin{equation}
\frac{d^2\theta}{dt^2}=-\frac{g}{l}\sin\theta
\label{eq:basic}
\end{equation}
%%%%%%%%%%%%%%
\subsection{線形の解}
角度$\theta$が十分に小さければ、
\begin{equation}
\frac{d^2\theta}{dt^2}=-\frac{g}{l}\theta
\end{equation}
となる。
定数$k$を
\begin{equation}
k^2=\frac{\theta_0^2}{4}
\end{equation}
とする。ただし初期の角度は$\theta_0$で、このときの速度はゼロとする。
振り子の周期$T$は、
\begin{equation}
T=2\pi\sqrt{\frac{l}{g}}
\end{equation}
で、角度$\theta$は
\begin{equation}
\theta=2k\sin\left\{\sqrt\frac{g}{l}\left(t+\frac{T}{4}\right)\right\}
\end{equation}
となる。
%%%%%%%%%%%%%%%%
\subsection{厳密解}
式\eqref{eq:basic}の厳密解は、定数$k$や周期$T$を
\begin{equation}
k^2=\frac12\left(1-\cos\theta_0\right)
\end{equation}
および
\begin{equation}
T=4\sqrt{\frac{l}{g}}K(k)
\end{equation}
として
\begin{equation}
\theta=2\sin^{-1}\left[k \mbox{sn}\left\{\sqrt{\frac{g}{l}}\left(t+\frac{T}{4}\right),k\right\}\right]
\end{equation}
となる。ここで$K(k)$は、母数$k$の第1種完全楕円積分であり、$\mbox{sn}(u,k)$は、母数$k$におけるヤコビのsn楕円関数である。
\section{シミュレーション}
\end{document}