  \documentclass[a4paper,11pt]{jbook}
\usepackage{graphicx}% Include figure files
\usepackage{dcolumn}% Align table columns on decimal point
\usepackage{bm}% bold math
\usepackage{amssymb}
\usepackage{amsmath}
%\usepackage{here}
\def\vecR {\bm {\mathcal {R} } }
\def\R  {\mathcal {R} }
\pagestyle{plain}
\title{磁石振り子シミュレータ\\PEM}
\author{牧野真人}
\date{\today}
\begin{document}
\maketitle
\tableofcontents 
 %%%%%%%%%%%%%%%%%%%%%%%%%
\chapter{はじめに}
シミュレーションエンジンPEMについての説明をする。
PEMは力学に関してのシミュレーションを行う。
力学のシミュレーションであるから、力$F$を受ける質量$m$の物体が加速度$a$で運動するニュートンの運動方程式
\begin{equation}
ma=F
\end{equation}
が基礎になる。
しかし、PEMでは、回転の運動方程式を中心に解く。
コマ、振り子のように、一様重力場中で一固定点を持った剛体の回転の問題を解く。
特に、剛体は、磁石を持つとして、磁気ダイポールをもっており、外磁場やダイポール同士で相互作用する。

剛体の運動は、オイラー角あるいは、四元数を用いて計算されることが多いがPEMでは、粒子固定の直交座標系$\bm{u}_1,\bm{u}_2,\bm{u}_3$を計算していく。
厳密解を解くなどの場合は、オイラー角は有用であるし、四元数は、分子動力学シミュレーションのような多数の多体問題を解く場合は効率が高い。
一方で、私の感覚であるが、粒子固定の直交座標系で計算する場合は、分かりやすい。
そのため、ここでは、粒子固定の直交座標系を用いる。

さらに、他に見られない特徴として回転微分演算子
\begin{equation}
\vecR =\sum_{i=1,2,3}\bm{u}_i\times\frac{\partial}{\partial \bm{u}_i}
\end{equation}
を用いる。
%%%%%%%%%%%%%%%%%%%%%%%%%%%%%%%%%%
\chapter{記号など}

%%%%%%%%%%%%%%%%%%%%%%%%%%%%%%%%%%
\chapter{理論}
\section{基礎方程式}
実験室が基底ベクトル$\bm{e}_x, \bm{e}_y,\bm{e}_z $で記述される空間とする。
ここで、$\bm{e}_i\cdot\bm{e}_j=\delta_{ij}, (i,j=x,y,z)$である。右手系として$\bm{e}_i\times\bm{e}_j=e_{ijk}\bm{e}_k$である。
ある一体の剛体を考える。この剛体は剛体に固定された基底ベクトル$\bm{u}_1,\bm{u}_2,\bm{u}_3$で剛体の方向を定義する。
この場合も$\bm{u}_i\cdot\bm{u}_j=\delta_{ij}, (i,j=1,2,3)$である。
トルク$\bm{T}$が与えられた際、角運動量$\bm{L}$の時間微分で与えられる。
\begin{equation}
\frac{d\bm{L}}{dt}=\bm{T}
\end{equation}
また、剛体の角速度$\bm{\omega}$は、剛体の慣性モーメントテンソル$\bm{I}$として
\begin{equation}
\bm{I}\cdot\bm{\omega}=\bm{L}
\end{equation}
となる。
慣性モーメントテンソルは、時間に応じて変化する。
しかし、慣性モーメントテンソルはあらかじめ、粒子に固定した座標系で計算するほうが便利である。
そのため、慣性モーメントテンソルは、次のようにする。
\begin{equation}
\bm{I}(t)=\sum_{i,j=1,2,3}I_{ij}\bm{u}_i(t)\bm{u}_j(t)
\end{equation}
$I_{ij}$は時間に依存しない定数である。粒子固定の座標系で最初に求めておけばよい。
一方で、行列$I_{ij}$の逆行列を$(I^{-1})_{ij}$とすると角速度は
 \begin{equation}
 \bm{\omega}=\sum_{i,j=1,2,3}(I^{-1})_{ij}\bm{L}\cdot\bm{u}_i\bm{u}_j
 \end{equation}
 となる。
 これから、
 %%%%%%%%%%%%%%%%%%%%%%%%%%%%%%%%%%%%%%%
\chapter{シミュレーション方法}
 %%%%%%%%%%%%%%%%%%%%%%%%%%%%%%%%
\chapter{}
 
  \end{document}

